\documentclass[professionalfont]{beamer}

\usepackage[T1]{fontenc}
\usepackage{amsmath}
\usepackage{graphicx}
\graphicspath{{figures/}}
\usepackage{tikz}
\usepackage{textpos}
 
% usage: \tikzpic{<x percent>}{<y percent>}{<image size>}{<image name>}
\newcommand*{\tikzpic}[4]{%
\begin{tikzpicture}[remember picture,overlay]
\node at (current page.south west) [xshift=#1\paperwidth,yshift=#2\paperheight] {\includegraphics[width=#3\linewidth]{#4}};
\end{tikzpicture}%
}

% Utility commands for programmatic multiple slide numbering
\newcounter{slideCounter}
\newcommand{\resetSlideCounter}{\setcounter{slideCounter}{0}}
\resetSlideCounter

% Programmatic slide naming
\newcommand{\titlesec}{\frametitle{\textbf{\insertsectionhead}}}
\newcommand{\titlesecnum}{\frametitle{\textbf{\insertsectionhead} \;(\arabic{slideCounter})}
\stepcounter{slideCounter}}
\newcommand{\titlesub}{\frametitle{\textbf{\insertsectionhead}\;|\; \insertsubsectionhead}}
\newcommand{\titlesubnum}{
\frametitle{\textbf{\insertsectionhead}\;|\; \insertsubsectionhead \;(\arabic{slideCounter})}
\stepcounter{slideCounter}}

% Command for highlighting text
\newcommand{\hgl}[1]{\textcolor{POLIMIblue}{\textbf{#1}}}

% Enumeration on multiple slides
\newcounter{saveenumi}
\newcommand{\seti}{\setcounter{saveenumi}{\value{enumi}}}
\newcommand{\conti}{\setcounter{enumi}{\value{saveenumi}}}
\resetcounteronoverlays{saveenumi}




\title[Social conversational spaces]%
{\textbf{Designing social conversational spaces}:\\enhancing communicative and emotional impact}
\author[L. Spagnolo]{Luigi Spagnolo\\
{\small \href{mailto:spagnolo@elet.polimi.it}{\nolinkurl{spagnolo@elet.polimi.it}}}}
\date{\today}
\institute[POLIMI]{\textbf{Politecnico di Milano}\\Department of Electronics and Information}

\usetheme{POLIMI}

\begin{document}
\begin{frame}
\titlepage
\end{frame}

\AtBeginSection[]
{
  \begin{frame}
    \frametitle{Outline}
    %\tableofcontents[currentsection,hideallsubsections]
	  \tableofcontents[currentsection]
  \end{frame}
}

\section{Example of section}
\subsection{Open questions}
\begin{frame}
\titlesub
\begin{itemize}
\item What does mean that the information contained in a given conversation is \emph{interesting} or \emph{relevant}? For which users?  In which context?
\item Which measurable features of social discussions should be considered? 
\item What are the best strategies for recommending or highlighting the most interesting contributions?
\item What are the best strategies for exploring on-going and past discussions?
\item How to handle large social spaces?
\begin{itemize} 
\item Many users with different background and expertise (e.g in education and cultural heritage)
\item An high volume of messages exchanged (blogs with thousands of comments per post
\end{itemize} 
\end{itemize} 
\end{frame}

\subsection{My research work}
\begin{frame}
\titlesub
\begin{block}{Communication-oriented approach}
Investigating possible relevant features for providing users with a better \hgl{sense making} of
\begin{itemize}
\item the \hgl{themes} of the discussion
\item the degree of \hgl{quality} and \hgl{interestingness} of a conversation
\item the \hgl{authoritativeness} and degree of \hgl{involvement} of its \hgl{participants}. 
\end{itemize} 
\end{block}
Towards a framework of \hgl{design strategies} for 
\begin{itemize}
\item determining the relevant features of discussions from metadata extracted using standard \hgl{text and opinion mining} tools
\item \hgl{visual representation} of \hgl{past and ongoing conversations}, supporting
\begin{itemize}
\item at a glance understanding of themes and participants involved
\item deeper investigation, analysis and retrieval
\end{itemize} 
\end{itemize} 
\end{frame}

\section{Another section}
\resetSlideCounter
\subsection{The subsection}
\begin{frame}
\titlesubnum
\begin{itemize}
\item First
\item Second
\end{itemize}
\end{frame}
\begin{frame}
\titlesubnum
\begin{itemize}
\item Third
\item Fourth
\end{itemize}
\end{frame}

\resetSlideCounter
\subsection{Another subsection}
\begin{frame}
\titlesubnum
\POLIMItitle{POLIMI Title}
\begin{enumerate} [i)]
\item First
\item Second
\item Third
\seti
\end{enumerate}
\end{frame}
\begin{frame}
\titlesubnum
\begin{enumerate} [i)]
\conti
\item Fourth
\item Fifth
\end{enumerate}
\end{frame}


\end{document}